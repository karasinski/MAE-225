\documentclass[onecolumn,10pt]{jhwhw}

\usepackage{algpseudocode}
\usepackage{amsfonts}
\usepackage{amsmath}
\usepackage{bm}
\usepackage{booktabs}
\usepackage{caption}
\usepackage{color}
\usepackage{commath}
\usepackage{empheq}
\usepackage{epsfig}
\usepackage{framed}
\usepackage{graphicx}
\usepackage{grffile}
\usepackage{listings}
\usepackage{mathtools}
\usepackage{pdfpages}
\usepackage{pgfplots}
\usepackage{siunitx}
\usepackage{wrapfig}

% Command "alignedbox{}{}" for a box within an align environment
% Source: http://www.latex-community.org/forum/viewtopic.php?f=46&t=8144
\newlength\dlf  % Define a new measure, dlf
\newcommand\alignedbox[2]{
% Argument #1 = before & if there were no box (lhs)
% Argument #2 = after & if there were no box (rhs)
&  % Alignment sign of the line
{
\settowidth\dlf{$\displaystyle #1$}
    % The width of \dlf is the width of the lhs, with a displaystyle font
\addtolength\dlf{\fboxsep+\fboxrule}
    % Add to it the distance to the box, and the width of the line of the box
\hspace{-\dlf}
    % Move everything dlf units to the left, so that & #1 #2 is aligned under #1 & #2
\boxed{#1 #2}
    % Put a box around lhs and rhs
}
}
% Default fixed font does not support bold face
\DeclareFixedFont{\ttb}{T1}{txtt}{bx}{n}{12} % for bold
\DeclareFixedFont{\ttm}{T1}{txtt}{m}{n}{12}  % for normal

\def\du#1{\underline{\underline{#1}}}

\author{John Karasinski}
\title{Project}

\begin{document}
%\maketitle

\begin{enumerate}
  \item Introduction
    \begin{enumerate}
      \item What is it? Diagram!
    \end{enumerate}
  \item Finite Kinematic Analysis
    \begin{enumerate}
      \item Method 1: D-H Formulation
        \begin{enumerate}
          \item Derive kinematic equation using D-H matrices, present Table of D-H parameters
        \end{enumerate}
      \item Method 2: Derive Shape/Joint Matrices
        \begin{enumerate}
          \item Solution for the direct and inverse kinematic
        \end{enumerate}
      \item Numerical Example
    \end{enumerate}
  \item Differential Kinematic Analysis
    \begin{enumerate}
      \item Derive the Jacobian using two methods
      \item Write velocity equation
    \end{enumerate}
  \item Conclusions
  \item References
\end{enumerate}

\section{Introduction}
We're going to play with a shoulder roll locked SSRMS.

\section{Finite Kinematic Analysis}
\subsection{Denavit-Hartenberg Parameters}

\begin{table}[h]
\centering
\begin{tabular}{*{5}{c}}
\toprule
$i$ & $\theta_i$ & $\alpha_i$ & $a_i$ & $d_i$ \\
\midrule
1 & $\theta_1$ & 90 &     0 & $d_1$ \\
2 & $\theta_2$ & 90 &     0 & $d_2$ \\
3 & $\theta_3$ &  0 & $a_3$ & $d_3$ \\
4 & $\theta_4$ &  0 & $a_4$ &     0 \\
5 & $\theta_5$ & 90 &     0 &     0 \\
6 & $\theta_6$ & 90 &     0 & $d_6$ \\
7 & $\theta_7$ & 90 &     0 & $d_7$ \\
\bottomrule
\end{tabular}
\caption{The Denavit-Hartenberg parameters for the SSRMS. These parameters are the joint angle, $\theta$, the link twist angle, $\alpha$, the link length, $a$, and the joint offset, $d$.}
\end{table}

\subsection{Inverse Kinematics Solution}
In general we can define
\begin{align*}
T_{07} &=
\left[\begin{matrix}
n_x & o_x & a_x & p_x \\
n_y & o_y & a_y & p_y \\
n_z & o_z & a_z & p_z \\
  0 &   0 &   0 &   1 \\
\end{matrix}\right]\\
&= T_{01} T_{12} T_{23} T_{34} T_{45} T_{56} T_{67}
\end{align*}
Premultiplying both sides by $T_{01}^{-1}$ yields,
\begin{align*}
T_{01}^{-1} T_{07} = T_{12} T_{23} T_{34} T_{45} T_{56} T_{67}
\end{align*}
Equating each element $(i,j)$ on both the left and right hand sides yields:
\begin{align*}
(1,1) \hspace{1em} & n_{x} c_1 + n_{y} s_1 &&= \left(s_{2} s_{6} + c_{2} c_{6} c_{345}\right) c_{7} + s_{7} s_{345} c_{2} \\
(1,2) \hspace{1em} & o_{x} c_1 + o_{y} s_1 &&= - s_{2} c_{6} + s_{6} c_{2} c_{345} \\
(1,3) \hspace{1em} & a_{x} c_1 + a_{y} s_1 &&= \left(s_{2} s_{6} + c_{2} c_{6} c_{345}\right) s_{7} - s_{345} c_{2} c_{7} \\
(1,4) \hspace{1em} & p_{x} c_1 + p_{y} s_1 &&= a_{3} c_{2} c_{3} + a_{4} c_{2} c_{34} + d_{3} s_{2} + d_{6} s_{345} c_{2} - d_{7} s_{2} c_{6} + d_{7} s_{6} c_{2} c_{345}\\
(2,1) \hspace{1em} & n_{z}                 &&= \left(s_{2} c_{6} c_{345} - s_{6} c_{2}\right) c_{7} + s_{2} s_{7} s_{345} \\
(2,2) \hspace{1em} & o_{z}                 &&= s_{2} s_{6} c_{345} + c_{2} c_{6} \\
(2,3) \hspace{1em} & a_{z}                 &&= \left(s_{2} c_{6} c_{345} - s_{6} c_{2}\right) s_{7} - s_{2} s_{345} c_{7} \\
(2,4) \hspace{1em} & - d_{1} + p_{z}       &&= a_{3} s_{2} c_{3} + a_{4} s_{2} c_{34} - d_{3} c_{2} + d_{6} s_{2} s_{345} + d_{7} s_{2} s_{6} c_{345} + d_{7} c_{2} c_{6}\\
(3,1) \hspace{1em} & n_{x} s_1 - n_{y} c_1 &&= - s_{7} c_{345} + s_{345} c_{6} c_{7} \\
(3,2) \hspace{1em} & o_{x} s_1 - o_{y} c_1 &&= s_{6} s_{345} \\
(3,3) \hspace{1em} & a_{x} s_1 - a_{y} c_1 &&= s_{7} s_{345} c_{6} + c_{7} c_{345} \\
(3,4) \hspace{1em} & p_{x} s_1 - p_{y} c_1 &&= a_{3} s_{3} + a_{4} s_{34} + d_{2} - d_{6} c_{345} + d_{7} s_{6} s_{345}\\
(4,1) \hspace{1em} & 0                     &&= 0 \\
(4,2) \hspace{1em} & 0                     &&= 0 \\
(4,3) \hspace{1em} & 0                     &&= 0 \\
(4,4) \hspace{1em} & 1                     &&= 1\\
\end{align*}
where I have defined $s_i = \sin{i}, c_i = \cos{i}, s_{ij} = \sin{i+j}, c_{ij} = \cos{i+j}, s_{ijk} = \sin{i+j+k}$ and $c_{ijk} = \cos{i+j+k}$.


\end{document}
