\documentclass[onecolumn,10pt]{jhwhw}

\usepackage{algpseudocode}
\usepackage{amsfonts}
\usepackage{amsmath}
\usepackage{bm}
\usepackage{booktabs}
\usepackage{caption}
\usepackage{color}
\usepackage{commath}
\usepackage{empheq}
\usepackage{epsfig}
\usepackage{framed}
\usepackage{graphicx}
\usepackage{grffile}
\usepackage{listings}
\usepackage{mathtools}
\usepackage{pdfpages}
\usepackage{pgfplots}
\usepackage{siunitx}
\usepackage{wrapfig}

% Command "alignedbox{}{}" for a box within an align environment
% Source: http://www.latex-community.org/forum/viewtopic.php?f=46&t=8144
\newlength\dlf  % Define a new measure, dlf
\newcommand\alignedbox[2]{
% Argument #1 = before & if there were no box (lhs)
% Argument #2 = after & if there were no box (rhs)
&  % Alignment sign of the line
{
\settowidth\dlf{$\displaystyle #1$}
    % The width of \dlf is the width of the lhs, with a displaystyle font
\addtolength\dlf{\fboxsep+\fboxrule}
    % Add to it the distance to the box, and the width of the line of the box
\hspace{-\dlf}
    % Move everything dlf units to the left, so that & #1 #2 is aligned under #1 & #2
\boxed{#1 #2}
    % Put a box around lhs and rhs
}
}
% Default fixed font does not support bold face
\DeclareFixedFont{\ttb}{T1}{txtt}{bx}{n}{12} % for bold
\DeclareFixedFont{\ttm}{T1}{txtt}{m}{n}{12}  % for normal

\def\du#1{\underline{\underline{#1}}}

\author{John Karasinski}
\title{Project}

\begin{document}
%\maketitle

\begin{enumerate}
  \item Introduction
    \begin{enumerate}
      \item What is it? Diagram!
    \end{enumerate}
  \item Finite Kinematic Analysis
    \begin{enumerate}
      \item Method 1: D-H Formulation
        \begin{enumerate}
          \item Derive kinematic equation using D-H matrices, present Table of D-H parameters
        \end{enumerate}
      \item Method 2: Derive Shape/Joint Matrices
        \begin{enumerate}
          \item Solution for the direct and inverse kinematic
        \end{enumerate}
      \item Numerical Example
    \end{enumerate}
  \item Differential Kinematic Analysis
    \begin{enumerate}
      \item Derive the Jacobian using two methods
      \item Write velocity equation
    \end{enumerate}
  \item Conclusions
  \item References
\end{enumerate}

\end{document}
