\documentclass[onecolumn,10pt]{jhwhw}

\usepackage{algpseudocode}
\usepackage{amsfonts}
\usepackage{amsmath}
\usepackage{bm}
\usepackage{booktabs}
\usepackage{caption}
\usepackage{color}
\usepackage{commath}
\usepackage{empheq}
\usepackage{epsfig}
\usepackage{framed}
\usepackage{graphicx}
\usepackage{grffile}
\usepackage{listings}
\usepackage{mathtools}
\usepackage{pdfpages}
\usepackage{pgfplots}
\usepackage{siunitx}
\usepackage{wrapfig}

% Command "alignedbox{}{}" for a box within an align environment
% Source: http://www.latex-community.org/forum/viewtopic.php?f=46&t=8144
\newlength\dlf  % Define a new measure, dlf
\newcommand\alignedbox[2]{
% Argument #1 = before & if there were no box (lhs)
% Argument #2 = after & if there were no box (rhs)
&  % Alignment sign of the line
{
\settowidth\dlf{$\displaystyle #1$}
    % The width of \dlf is the width of the lhs, with a displaystyle font
\addtolength\dlf{\fboxsep+\fboxrule}
    % Add to it the distance to the box, and the width of the line of the box
\hspace{-\dlf}
    % Move everything dlf units to the left, so that & #1 #2 is aligned under #1 & #2
\boxed{#1 #2}
    % Put a box around lhs and rhs
}
}
% Default fixed font does not support bold face
\DeclareFixedFont{\ttb}{T1}{txtt}{bx}{n}{12} % for bold
\DeclareFixedFont{\ttm}{T1}{txtt}{m}{n}{12}  % for normal

\def\du#1{\underline{\underline{#1}}}

\author{John Karasinski}
\title{Homework \# 1}

\begin{document}
%\maketitle

\problem{}
In an industrial application a part is to turn 20 degrees about a rod shown below (in the direction indicated) and move down six inches along the same rod.
\begin{enumerate}
    \item Please determine a single 4x4 matrix transformation that can be used to compute the new coordinates of an arbitrary point on the part.
    \item Using the results of part a, please determine the new world coordinates of a point whose original coordinates were X, Y, Z.
\end{enumerate}

Given
\begin{align*}
P =
\begin{bmatrix*}[c]
0 \\
0 \\
6
\end{bmatrix*}
O &=
\begin{bmatrix*}[c]
0 \\
8 \\
0
\end{bmatrix*},\\
w = \dfrac{P - O}{\norm{P - O}} &=
\begin{bmatrix*}[c]
0 \\
-.8 \\
.6
\end{bmatrix*}
\end{align*}

\begin{align*}
\du{\Theta} &= \du{I} + \du{\widetilde{w}} \sin \phi + \du{\widetilde{w}}^2 \left ( 1 - \cos\phi \right ) \\
&=
\begin{bmatrix*}[c]
(w_x^2-1)(1 -\cos \theta) + 1 & w_xw_y(1 -\cos \theta) - w_z\sin \theta &  w_xw_z(1 -\cos \theta) + w_y\sin \theta \\
w_xw_y(1 -\cos \theta) + w_z\sin \theta &  (w_y^2-1)(1 -\cos \theta) + 1 & w_yw_z(1 -\cos \theta)-w_x\sin \theta  \\
w_xw_z(1 -\cos \theta) - w_y\sin \theta & w_yw_z(1 -\cos \theta) + w_x\sin \theta &   (w_z^2-1)(1 -\cos \theta) + 1 \\
\end{bmatrix*} \\
&=
\begin{bmatrix*}[c]
0.9396 & -0.2052 & -0.2736 \\
0.2052 &  0.9782 & -0.0289 \\
0.2736 & -0.0289 &  0.9614 \\
\end{bmatrix*} \\
d &= \left(\du{I} - \du{\Theta} \right ) \underline{P} + \phi \underline{w} \\
&=
\begin{bmatrix*}[c]
\phi w_x   -P_x (w_x^2-1) (1-\cos \theta) + P_y (-w_x w_y (1-\cos \theta) + w_z \sin \theta) - Pz (w_x w_z (1-\cos \theta) + w_y \sin \theta)\\
\phi w_y -  P_x ( w_x w_y (1-\cos \theta) + w_z \sin \theta) - P_y (w_y^2 - 1) (1-\cos \theta) + Pz (- w_y w_z (1-\cos \theta)+w_x \sin \theta )\\
\phi w_z +  P_x (-w_x w_z (1-\cos \theta) + w_y \sin \theta) - P_y (w_y w_z (1-\cos \theta) + w_x \sin \theta ) - Pz (w_z^2 - 1) (1-\cos \theta)\\
\end{bmatrix*} \\
&=
\begin{bmatrix*}[c]
 1.6416 \\
-4.6263 \\
 3.8315 \\
\end{bmatrix*} \\
T &=
\begin{bmatrix*}[c]
0.9396 & -0.2052 & -0.2736 &  1.6416 \\
0.2052 &  0.9782 & -0.0289 & -4.6263 \\
0.2736 & -0.0289 &  0.9614 &  3.8315 \\
0 & 0 & 0 & 1
\end{bmatrix*}
\end{align*}

\begin{align*}
T
\begin{bmatrix*}[c]
X \\
Y \\
Z \\
1 \\
\end{bmatrix*}
=
\begin{bmatrix*}[c]
0.9396 X -  0.2052 Y -  0.2736 Z +  1.6416 \\
0.2052 X +  0.9782 Y -  0.0289 Z -  4.6263 \\
0.2736 X -  0.0289 Y +  0.9614 Z +  3.8315 \\
                                         1 \\
\end{bmatrix*}
\end{align*}

\problem{}
Complete the derivation of the Denavit-Hartenberg (D-H) Transformation from what was done in the classroom using the joint and Shape matrices based on the diagram and the derivation started in the class on 2-1-2018.

\begin{align*}
\du{T}_{h-, h+} &= \du{I} \Phi_{h} \du{S}^{-1}_{h+1, h} \\
\Phi_{h} &=
\begin{bmatrix*}[c]
\cos \phi_h^1 & -\sin \phi_h^1 & 0 & 0 \\
\sin \phi_h^1 &  \cos \phi_h^1 & 0 & 0 \\
0 & 0 & 1 & \phi_h^2 \\
0 & 0 & 0 & 1
\end{bmatrix*} \\
\du{\Theta} &= \du{I} + \du{\widetilde{w}} \sin \phi + \du{\widetilde{w}}^2 \left ( 1 - \cos\phi \right ) \\
\du{\widetilde{w}} &=
\begin{bmatrix*}[c]
0 & -w_z & w_y \\
w_z & 0 & -w_z \\
-w_y & w_x & 0
\end{bmatrix*}
\end{align*}
To solve for $\du{S}^{-1}_{h+1, h}$, we define the transformation: \\
\\
Fixed coordinate system: $(u v w)_{h'}$ \\
Moving coordinate system: $(u v w)_{h+1}$ \\
Screw axis: $u'_h, w = (1, 0, 0)$ \\
$\underline{P} = \underline{0}$ \\
$s = a_h$, $u$ direction \\
$\phi = \alpha_h$

\begin{align*}
\du{\Theta} &=
\begin{bmatrix*}[c]
1 & 0 & 0 \\
0 & 1 & 0 \\
0 & 0 & 1 \\
\end{bmatrix*} +
\begin{bmatrix*}[c]
0 & 0 & 0 \\
0 & 0 & -1 \\
0 & 1 & 0
\end{bmatrix*} \sin \alpha_h +
\begin{bmatrix*}[c]
0 & 0 & 0 \\
0 & 0 & -1 \\
0 & 1 & 0
\end{bmatrix*}
\begin{bmatrix*}[c]
0 & 0 & 0 \\
0 & 0 & -1 \\
0 & 1 & 0
\end{bmatrix*} \left ( 1 - \cos \alpha_h \right ) \\
&=
\begin{bmatrix*}[c]
1 &        0 &         0 \\
0 & \cos \alpha_h & -\sin \alpha_h \\
0 & \sin \alpha_h &  \cos \alpha_h \\
\end{bmatrix*} \\
S_{h+,h}^{-1} &=
\begin{bmatrix*}[c]
1 &        0 &         0  & a_h \\
0 & \cos \alpha_h & -\sin \alpha_h & 0 \\
0 & \sin \alpha_h &  \cos \alpha_h & 0\\
0 & 0 & 0 & 1
\end{bmatrix*} \\
\du{T}_{h-, h+} &= \du{I} \Phi_{h} \du{S}^{-1}_{h+1, h} \\
&=
\begin{bmatrix*}[c]
\cos \phi_h^1 & -\sin \phi_h^1 & 0 & 0 \\
\sin \phi_h^1 &  \cos \phi_h^1 & 0 & 0 \\
0 & 0 & 1 & \phi_h^2 \\
0 & 0 & 0 & 1
\end{bmatrix*}
\begin{bmatrix*}[c]
1 &        0 &         0  & a_h \\
0 & \cos \alpha_h & -\sin \alpha_h & 0 \\
0 & \sin \alpha_h &  \cos \alpha_h & 0\\
0 & 0 & 0 & 1 \\
\end{bmatrix*} \\
&=
\begin{bmatrix*}[c]
\cos \phi_h^1  &  -\sin \phi_h^1  \cos \alpha_h  &  \sin \alpha_h  \sin \phi_h^1  & a_h \cos \phi_h^1  \\
\sin \phi_h^1  &   \cos \alpha_h  \cos \phi_h^1  & -\sin \alpha_h  \cos \phi_h^1  & a_h \sin \phi_h^1  \\
             0 &                  \sin \alpha_h  &                 \cos \alpha_h &           \phi_h^2 \\
             0 &                               0 &                              0 &                  1 \\
\end{bmatrix*}
\end{align*}

\problem{}
Consider the robot manipulator shown below.
\begin{enumerate}
    \item Determine the D-H parameters for the robot and the D-H transformation for each joint.
    \item Derive the kinematic equations for the coordinates of a point at the tip of the last link (XYZ) in terms of the joint variables.
    \item Determine the inverse kinematic solution.
\end{enumerate}

\begin{center}
\begin{tabular}{r|rrr}
           & 1 & 2 & 3 \\
\midrule
$a_i$      &          0 &   0 &   0 \\
$\alpha_i$ &          0 & -90 &   0 \\
$\theta_i$ &   $\theta$ &  90 &   0 \\
$s_i$      &          h &   0 &   r \\
\end{tabular}
\end{center}

\begin{align*}
\du{T}_{12} &=
\begin{bmatrix*}[c]
\cos\theta & -\sin\theta & 0 & 0 \\
\sin\theta &  \cos\theta & 0 & 0 \\
         0 &           0 & 1 & h \\
         0 &           0 & 0 & 1 \\
\end{bmatrix*},
\du{T}_{23} =
\begin{bmatrix*}[c]
0 &  0 & -1 & 0 \\
1 &  0 &  0 & 0 \\
0 & -1 &  0 & 0 \\
0 &  0 &  0 & 1 \\
\end{bmatrix*},
\du{T}_{34} =
\begin{bmatrix*}[c]
1 & 0 & 0 & 0 \\
0 & 1 & 0 & 0 \\
0 & 0 & 1 & r \\
0 & 0 & 0 & 1 \\
\end{bmatrix*}
\end{align*}
\begin{align*}
\underline{x}_1 &= \du{T}_{12} \du{T}_{23} \du{T}_{34} \underline{x}_4 \\
\begin{bmatrix*}[c]
x_1 \\
y_1 \\
z_1 \\
1 \\
\end{bmatrix*}
& =
\begin{bmatrix*}[c]
-\sin \theta &  0 & -\cos \theta & -r \cos \theta \\
 \cos \theta &  0 & -\sin \theta & -r \sin \theta \\
          0 & -1 &           0 &             h \\
          0 &  0 &           0 &             1 \\
\end{bmatrix*}
\begin{bmatrix*}[c]
X \\
Y \\
Z \\
1 \\
\end{bmatrix*}\\
&=
\begin{bmatrix*}[c]
-X \sin \theta - Z \cos \theta - r \cos \theta \\
 X \cos \theta - Z \sin \theta - r \sin \theta \\
                                      -Y + h \\
                                            1 \\
\end{bmatrix*}\\
% \underline{x}_4 &= \left ( \du{T}_{12} \du{T}_{23} \du{T}_{34} \right )^{-1} = \du{T}_{41} \underline{x}_1 \\
% \begin{bmatrix*}[c]
% X \\
% Y \\
% Z \\
% 1 \\
% \end{bmatrix*}
% &=
% \begin{bmatrix*}[c]
%  -\sin\theta &  \cos\theta &  0 &  0 \\
%            0 &           0 & -1 &  h \\
%  -\cos\theta & -\sin\theta &  0 & -r \\
%            0 &           0 &  0 &  1 \\
% \end{bmatrix*}
% \begin{bmatrix*}[c]
% x_1 \\
% y_1 \\
% z_1 \\
% 1 \\
% \end{bmatrix*} \\
% &=
% \begin{bmatrix*}[c]
%     -x_1 \sin \theta + y_1 \cos \theta \\
%                               -z_1 + h \\
% -x_1 \cos \theta - y_1 \sin \theta - r \\
%                                      1 \\
% \end{bmatrix*}
\end{align*}
Looking at the last column of the transformation matrix, we can write
\begin{align*}
\alignedbox{X}{= -r \cos \theta} \\
\alignedbox{Y}{= -r \sin \theta} \\
\alignedbox{Z}{= h} \\
\end{align*}
From which we can immediately pull
\begin{align*}
\alignedbox{h}{=Z}
\end{align*}
And, with some minor manipulation,
\begin{align*}
X^2 &= r^2 \cos^2 \theta \\
Y^2 &= r^2 \sin^2 \theta \\
\alignedbox{r}{= (X^2 + Y^2)^{\frac{1}{2}}} \\
\dfrac{X}{(X^2 + Y^2)^{\frac{1}{2}}} &= -\cos \theta \\
\alignedbox{\theta}{=\cos^{-1} \left ( \dfrac{-X}{(X^2 + Y^2)^{\frac{1}{2}}} \right)}
\end{align*}

\end{document}
